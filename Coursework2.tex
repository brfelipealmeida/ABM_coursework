


% ------------------ DOCUMENT SETUP ------------------ 
% The document class defines the document type (report) and sets the font size (10pt)
\documentclass[10pt]{report}
\author{Felipe Santos Almeida}

% Inputs the Document Packages
\input{_Packages}

% Controls how many subsections the document can take
%  and how many of those will get put into the contents pages.
\setcounter{secnumdepth}{2}
\setcounter{tocdepth}{2}

% Line Spacing
\setstretch{1.0}

% The folder path where the images will be uploaded
\graphicspath{{./Images/}} 

% Places a dot after Chapter/Section/Subsection number in Table of Contents
\renewcommand{\cftchapaftersnum}{.}
\renewcommand{\cftsecaftersnum}{.}
\renewcommand{\cftsubsecaftersnum}{.}

%  Customize Dot spacing in Table of Contents/List of Figures/Tables
\renewcommand{\cftdotsep}{0.3}

% Numeration Type for Chapters and Sections (Roman I, II, II / Arabic 1, 2, 3)
\renewcommand\thechapter{\Roman{chapter}}
\renewcommand\thesection{\arabic{section}}

% Line Break Properties
\tolerance=1
\emergencystretch=\maxdimen
\hyphenpenalty=10000
\hbadness=10000


% Formatting Table of Contents/Lists titles
\renewcommand{\contentsname}{\normalfont\bfseries\LARGE{CONTENTS}}
\renewcommand{\listfigurename}{\normalfont\bfseries\LARGE{LIST OF FIGURES}}
\renewcommand{\listtablename}{\normalfont\bfseries\LARGE{LIST OF TABLES}}


% Title Formatting customization
\titleformat{\chapter}{\normalfont\bfseries\LARGE}{\thechapter.}{1em}{\MakeUppercase}

\titleformat{\section}{\normalfont\bfseries\large}{\thesection.}{1em}{\MakeUppercase}
\titlespacing*{\section} {0pt} {0pt} {15pt} % left, before, after

\titleformat{\subsection}{\normalfont\bfseries\large}{\thesubsection.}{1em}{}
\titlespacing*{\subsection} {0pt} {10pt} {10pt}

\titleformat{\subsubsection}{\normalfont\bfseries\large}{}{1em}{}
\titlespacing*{\subsubsection} {0pt} {10pt} {10pt}


% HEADER AND FOOTER
\pagestyle{fancy}  % Set Page Style (Header and Footer Style)
\fancyhf{}  % Clears the header and footer (from the default info)

% Header
\renewcommand{\headrulewidth}{0pt}  % Removes the default Horizontal Line in Header
\fancyhead[L]{SN21125032}
\fancyhead[R]{May 2023}

% Footer
\fancyfoot[C]{\thepage} % Page Number

% Change figure numbering per section
\numberwithin{figure}{section}
\numberwithin{table}{section}



 \captionsetup{justification=centering}


%  -------------------------------------------------
%  --------- The document starts from here --------- 
%  -------------------------------------------------

\begin{document}

% -------------------  INTRODUCTION  ---------------------

\begin{center}
    % UCL IMAGE
    \vspace*{-3cm}
    \makebox[\textwidth]{\includegraphics[width=\paperwidth]{Images/UCL_LOGO_new.png}}
\end{center}   
    % Title
    {\LARGE\textbf{Coursework 2\\
    % Subtitle
    CASA0011 – Agent-Based Modelling\\}}
SN21125032 | Word count:

\vspace{5mm} %5mm vertical space
  
\section{Research Question}

\section{ODD Description}
\subsection{Purpose and patterns}

According to ..... This a serious problem related to the lack of street space for pedestrians, and how this affects the rate of injuries caused by car crashes. Models developed on Netlogo related to this field, such as 15min cities ( brief explanation), Town - Traffic and Crowd Simulation( brief explanation), Traffic grid( brief explanation) and Taxi Cabs( brief explanation) illustrate the potentials of micro-simulation related to this field.
Therefore, this model aims to test if a micro-simulation of a street intersection using pedestrian flows in crossing a street can explain the relationship between sidewalk dimension and waiting time to cross the street. In this model, it is applied...... However, this model does not intend to simulate a realistic street intersection with all the agents. The aim is to intend pedestrian behaviour in the context of commute flows and their relationship with the available space for a given sidewalk.  

Thus, the aim is to .....

The patterns, 

First, a high concentration of people close by pedestrian crossing
Second, alternation in the density of people in opposite regions of the intersection
Third, travessy time is not proportional to the concentration of people

\subsection{Entities, state variables, and scales}





\subsection{Process overview and scheduling}

\subsection{Design concepts}


\subsection{Input data}


\subsection{Submodels}

\section{Brief Methodology }



Limitations

Adding the flow of cars, cyclists and traffic lights would be relevant.

















% -------------------  BIBLIOGRAPHY ---------------------
\newpage
\printbibliography[title = {References}]
\addcontentsline{toc}{chapter}{References} % Adds References Section to Table of Contents

\vspace{5mm} %5mm vertical space

To access GitHub Repository \href{https://github.com/brfelipealmeida/LondonTubeNetwork}{Link}

\end{document}
%  -------------------------------------------------
%  --------- The document ends from here ----------- 
%  -------------------------------------------------