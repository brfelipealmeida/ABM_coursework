


% ------------------ DOCUMENT SETUP ------------------ 
% The document class defines the document type (report) and sets the font size (10pt)
\documentclass[10pt]{report}
\author{Felipe Santos Almeida}

% Inputs the Document Packages
\input{_Packages}

% Controls how many subsections the document can take
%  and how many of those will get put into the contents pages.
\setcounter{secnumdepth}{2}
\setcounter{tocdepth}{2}

% Line Spacing
\setstretch{1.0}

% The folder path where the images will be uploaded
\graphicspath{{./Images/}} 

% Places a dot after Chapter/Section/Subsection number in Table of Contents
\renewcommand{\cftchapaftersnum}{.}
\renewcommand{\cftsecaftersnum}{.}
\renewcommand{\cftsubsecaftersnum}{.}

%  Customize Dot spacing in Table of Contents/List of Figures/Tables
\renewcommand{\cftdotsep}{0.3}

% Numeration Type for Chapters and Sections (Roman I, II, II / Arabic 1, 2, 3)
\renewcommand\thechapter{\Roman{chapter}}
\renewcommand\thesection{\arabic{section}}

% Line Break Properties
\tolerance=1
\emergencystretch=\maxdimen
\hyphenpenalty=10000
\hbadness=10000


% Formatting Table of Contents/Lists titles
\renewcommand{\contentsname}{\normalfont\bfseries\LARGE{CONTENTS}}
\renewcommand{\listfigurename}{\normalfont\bfseries\LARGE{LIST OF FIGURES}}
\renewcommand{\listtablename}{\normalfont\bfseries\LARGE{LIST OF TABLES}}


% Title Formatting customization
\titleformat{\chapter}{\normalfont\bfseries\LARGE}{\thechapter.}{1em}{\MakeUppercase}

\titleformat{\section}{\normalfont\bfseries\large}{\thesection.}{1em}{\MakeUppercase}
\titlespacing*{\section} {0pt} {0pt} {15pt} % left, before, after

\titleformat{\subsection}{\normalfont\bfseries\large}{\thesubsection.}{1em}{}
\titlespacing*{\subsection} {0pt} {10pt} {10pt}

\titleformat{\subsubsection}{\normalfont\bfseries\large}{}{1em}{}
\titlespacing*{\subsubsection} {0pt} {10pt} {10pt}


% HEADER AND FOOTER
\pagestyle{fancy}  % Set Page Style (Header and Footer Style)
\fancyhf{}  % Clears the header and footer (from the default info)

% Header
\renewcommand{\headrulewidth}{0pt}  % Removes the default Horizontal Line in Header
\fancyhead[L]{SN21125032}
\fancyhead[R]{May 2023}

% Footer
\fancyfoot[C]{\thepage} % Page Number

% Change figure numbering per section
\numberwithin{figure}{section}
\numberwithin{table}{section}



 \captionsetup{justification=centering}


%  -------------------------------------------------
%  --------- The document starts from here --------- 
%  -------------------------------------------------

\begin{document}

% -------------------  INTRODUCTION  ---------------------

\begin{center}
    % UCL IMAGE
    \vspace*{-3cm}
    \makebox[\textwidth]{\includegraphics[width=\paperwidth]{Images/UCL_LOGO_new.png}}
\end{center}   
    % Title
    {\LARGE\textbf{Coursework 2\\
    % Subtitle
    CASA0011 – Agent-Based Modelling\\}}
SN21125032 | Word count: 1070

\vspace{5mm} %5mm vertical space
  
\section{Research Question}

\begin{enumerate}
    \item  What is the influence of the number of pedestrians, speed limit, and time to cross on the number of pedestrians waiting to cross a street?
    \item How can the number of people waiting to cross a street define the optimal space for sidewalks close to pedestrians crossing?
\end{enumerate}

\section{ODD Description}
\subsection{Purpose and patterns}

This model serves as a simplified micro-simulation of a street intersection to investigate the connection between pedestrians' spatial concentration and their waiting time to cross the street. It does not aim to replicate a realistic street intersection with all its agents but focuses on understanding pedestrian behaviour within commuting flows and the available sidewalk space.
The model explores three distinct patterns:
\begin{itemize}
    \item High concentration near pedestrian crossings
    \item Alternating density in opposite sidewalk regions
    \item Non-proportional crossing time
\end{itemize}

By examining these patterns, the model aims to understand how pedestrian behaviour, commute flows, and sidewalk availability are interconnected.
\subsection{Entities, state variables, and scales}

\textbf{Entities}

The entities in the project are the pedestrians who are walking in random directions, aiming to cross the street to continue the walk. In that context, the model also has the street, the sidewalk and the building representing the built environment. Besides that, it also has a pedestrian crossing and a separation in the middle of the streets.


\textbf{State variables}


\begin{table}[h]
\centering
\begin{tabular}{@{}lll@{}}
\toprule
\textbf{Variable name} & \textbf{Variable type and units} & \textbf{Meaning}                                                                                         \\ \midrule
meaning                & String                           & Represents the role of the patch                                                                         \\ \midrule
will-cross?            & Boolean                          & \begin{tabular}[c]{@{}l@{}}Indicates whether anybody is going \\ to cross the street\end{tabular}        \\ \midrule
used                   & Numeric                          & \begin{tabular}[c]{@{}l@{}}How many  pedestrians are \\ using the crossing\end{tabular}                  \\ \midrule
speed                  & Numeric                          & The current speed of a person                                                                            \\ \midrule
walk-time              & Numeric                          & \begin{tabular}[c]{@{}l@{}}How long the person will walk \\ before crossing the road\end{tabular}        \\ \midrule
crossing-part          & Numeric                          & Divides the crossing into parts                                                                          \\ \midrule
waiting?               & Boolean                          & \begin{tabular}[c]{@{}l@{}}Indicates whether the pedestrian is \\ waiting to cross the road\end{tabular} \\ \midrule
num-of-people          & Numeric                          & \begin{tabular}[c]{@{}l@{}}The desired number of people \\ in the simulation\end{tabular}                \\ \midrule
time-to-crossing       & Numeric                          & \begin{tabular}[c]{@{}l@{}}Represents the time taken \\ to cross the road\end{tabular}                   \\ \bottomrule
\end{tabular}
\caption{State variables}
\label{table: variables}
\end{table}

\newpage

\textbf{Scales}

The model has not a specified scale. 

\subsection{Process overview and scheduling}


The model is developed to simulate pedestrians' movement when they cross the street. The pedestrians follow specific rules such as walking on the sidewalk, waiting at the wait point close to the pedestrian crossing and then crossing the streets. Thus, the micro-simulation progress each interaction in the main loop following this step, allowing for tracking the movement and control of the interaction through the number of pedestrians, time to crossing and speed. To summarize, the simulations occur in that sequence:

1. Setup the environment
2. Main loop
3. Move pedestrians
4. Walking rules
5. Crossing the street
6. Simulation track


\subsection{Design concepts}

\subsubsection{Basic principles}

This model is based on the concept of agent pedestrian modelling in order to understand how ABM can support the understanding of pedestrian behaviour in the urban environment(\cite{batty_agent-based_2001}). Thus, this analysis is based on studies related to the simulation of pedestrians at intersections and their relationship with traffic accidents(\cite{cohen_microscopic_2018}), and also in order to understand pedestrian flow in a broader context(\cite{chen_delineating_2021}). To build this model, methods presented in the Town-Traffic \& Crowd simulation model(\cite{lukas_netlogo_nodate}) were applied to test the interaction of different agents in traffic.

\subsubsection{Emergence}
 Despite giving a range of rules for pedestrians' movements, the number of people waiting to cross the streets and the location of people flow is a consequence of the behaviour of agents in the given environment; the output of this micro-simulation depends on this interaction. 

\subsection{Initialization}

The code starts with the agents' definitions ( breed [crossings crossing] and breed [person person], representing the road crossings and pedestrians. And then, Is defines the Patches variables and person variables as mentioned before. Thus, 'to setup' initialise the simulation environment, creating the street network, crossings, and sidewalks, and placing the pedestrians in the environment. First, in 'draw-roads', this procedure creates the crossroads setting the patches in grey colour in four directions( road-up, road-down, road-left and road-right). 

Second, it created the sidewalks in 'draw-sidewalk', setting the patch colour in brown. After that is drawn the crossing on roads going up and down with a white square shape. The position is randomly positioned along the y-axis and x-axis. Besides that, it is implemented the wait points for pedestrians.

Third, this procedure places people on the sidewalks according to the desired number of people ('number-of-people). The model selects a random sidewalk patch and distributes the turtles with random attributes( speed, size, walk time, shape).



\subsection{Input data}

In the scenario presented, there are initially 75 people(0 to 1000 units). Crossing time ranges from 400 to 5000 units, with an initial value set at 1300 units. The speed limit for pedestrians is set between 30 and 150 units, with a default value of 50 units representing normal speed. The model's settings include a central location of origin, with a maximum x-coordinate and y-coordinate of 17 units each. The patch size is 10 units, and the font size for any displayed information is 10. The frame rate for visual updates is set at 30 frames per second.


\subsection{Submodels}

The submodels were built based on the model ( \cite{lukas_netlogo_nodate}) 

\subsubsection{move-people}
This procedure aims to move the person agents with the following rules:

\begin{itemize}
    \item Check if the Walk-time of a person is greater than or equal to the time-to-crossing
    \item Check if the person is at the crossing point( crossing-part>=1)
    \item Check if the agent is at a wait point, and then indicate the intention to cross.
    \item  Make the person close to the nearest wait point and then request a 'walk procedure' to move the agent
    \item  If walk-time is not sufficient for reaching the crossing, it calls the 'walk' procedure
\end{itemize}

\subsubsection{walk}
This submodel is responsible for the basic walking behaviour:

\begin{itemize}
    \item It checks the patch in front of the person's current position
    \item If the patch is a sidewalk or wait point, It checks if there is any other agent on that patch
    \item If the patch is not a sidewalk or wait point, the agent randomly turns right or left 
    \item At the end, it checks if the agent has reached a wait point and updates the attribute ( Waiting?)
\end{itemize}


\subsubsection{cross-the-street}

This procedure is focused on the agent's behaviour when crossing the street:
\begin{itemize}
    \item If the crossing part is 1, the agent faces the next wait point2 with a specific radius and sets the crossing part to 2
    \item If the crossing part is 2, the person adjusts its direction
    \item the agent set the crossing part from 3 to 0, indicating the crossing is complete, and then the walk time is reset to 0
    \item the agent moves forward by a fraction of their speed and then the waiting? Attribute change to false to indicate that it is no longer waiting.
\end{itemize}

\vspace{5mm} %5mm vertical space

\section{Brief Methodology }

The methodology employed in this analysis involves conducting an experiment using behaviour space. The objective is to examine the relationship between three key factors: the number of pedestrians, the speed limit imposed on pedestrians, and the time required to cross the street. The aim is to determine how these factors influence the number of individuals waiting before crossing the street. By conducting this experiment and analyzing the outcomes, the analysis seeks to establish parameters defining the critical aspects to consider in urban design, particularly street intersections. 


% -------------------  BIBLIOGRAPHY ---------------------
\printbibliography[title = {References}]
\addcontentsline{toc}{chapter}{References} % Adds References Section to Table of Contents

\vspace{5mm} %5mm vertical space

To access GitHub Repository \href{https://github.com/brfelipealmeida/ABM_coursework}{Link}

\end{document}
%  -------------------------------------------------
%  --------- The document ends from here ----------- 
%  -------------------------------------------------